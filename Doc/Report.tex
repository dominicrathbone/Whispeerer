\documentclass[]{report}
\usepackage{listings}

% Title Page
\title{WhisPeerer \\A WebRTC communications application}
\author{Dominic Rathbone \\ Student Number: 12843140 \\ Source Code \& Documentation: \\ https://github.com/domr115/CI360-Mobile-App-Dev}


\begin{document}
\maketitle

\tableofcontents

\chapter{Introduction}
The aim of this project was to provide an application enabling users to communicate using peer-to-peer technology. The motivation behind this was the increasing concern of security many consumers face in the modern age due to third parties, whether it be a corporation such as Facebook or an authority such as the UK Government, storing and analysing their data. By using a technology called WebRTC, it is possible for two users to form a direct peer to peer connection over the internet.  Although a server is used for the initial signalling and negotiation process between the peers, the communication data sent over this communications channel is never disclosed to an external server. On top of this, the application aimed to provide a sense of anonymity and impermanence by only providing users with a temporary, random unique identifier for a user-name and only storing data within the lifespan of the application.

	\chapter{Development}
		\section{Node.js Signalling Server}
			\subsection{API}
			In order to set up a space in which two users can exchange the meta-data needed for the negotiation of a peer connection, an API was created on the server using Express.js, a popular web application framework for Node.js. This consists of two endpoints, one endpoint to create a new user and one endpoint to check if the user exists. The former returns a unique, random session identifier to the application and the latter simply returns a status code of 200 or 404 dependent on whether the user exists or not. s. This API was designed using the Representation State Transfer (REST) architectural style in order to provide a uniform and consistent API to it's consumers. REST achieves this by modelling the endpoints in an API around the resources they represent with HTTP verbs representing the operations that are achievable on this resource. For example, in order to create a user on the server, the consumer send a POST request to a "/user" endpoint and in order to check if a user exists, the consumer sends a GET request to "/users/[userId]". This architectural style also makes the API more extensible as it is easy to design new endpoints.
			
			\subsection{WebSockets}
			To provide a bi-direction communications channels for the two application instances to negotiate the peer connection over, a WebSockets implementation called Socket.io was used. Socket.io models communication using the concepts of name spaces and rooms where a name space is represented by the endpoint a socket connects to and rooms are channels within this that users can join and leave. In this case, each user is represented by their own namespace "/user/[userId]". When another user wants to communicate with a user, they join their namespace, becoming a "guest". In turn, this puts it into a "busy" state preventing other users from joining. From this namespace, the two users can exchange messages and once they are done and the guest disconnects, taking it out of the "busy" state. By modelling the user's communications channels like this, the guest user can still receive incoming chat offers from other users whilst busy with another user, similar to how a mobile phone can still receive calls whilst in a call with another. 
			
		\section{Android Application}
			
			\subsection{Architecture}
			As the majority of the data processing is handled by the libraries, the application's responsibilities mainly lied in the triggering of and reaction to incoming messages from another instance of the application, whether it be through the signalling server or directly via the peer-to-peer connection. Due to this, the architecture of the application can be represented by an "Event-Driven Architecture" pattern. This is where the flow of control in an application is dictated by the emission of and reaction to "events" where an event is a change in the state of the application. 
			
			\subsection{Observer Patterns}
			To create the event-driven architecture, the application utilises the Observer pattern to trigger changes in the flow of the application by asynchronously updating activities (and thus the user interface) when certain events are received. The observer pattern is a design pattern in which 
			
			e, the majority of the "events" within the application take the form of JSON messages transmitted and received over the WebSockets connection with the server. To react to these socket events, the socket.io Java client library expects the application to provide an object implementing the "Listener" interface when setting up these events. This interface requires the object to implement methods with the code that is to be called when the event is received (see Appendix A for an example).
			
			On top of this, before the WebSockets connection is set up, the application communicates with the server via a HTTP API using a library called Android Asynchronous HTTP Client. This library makes it easy to send the requests to this API 
						
			In this case, 

 			Further to this, 
			
		This event-driven architecture is applied using several different design patterns.
		Observer Pattern
		Handler Pattern
		Factory Pattern
		\subsection{Android Async HTTP Client}
		\subsection{WebSockets}
		Socket.io client
		\subsection{WebRTC}
		\subsection{OpenGL}
		GLSurfaceView
		\subsection{Android Services}
		Camera
		Microphone
		Vibrator
		Share
	\section{External Technologies}
		\section{Git}
		\section{Android Studio}
		\section{WebStorm}
		
	\chapter{Design}
		As the application used relative new technologies, the majority of the project was focused on developing around these to ensure they worked. Due to this, the design aspect of the project was considered a lower priority.
		\section{Logo}
		\section{Color Scheme}
		\section{Layout}

	\chapter{Testing}
		\section{Automated Testing}
			\subsection{Unit Tests}
		\section{Manual Testing}
	
	\chapter{Reflection \& Review}
		\section{Background Research}
		\section{Methodology}
		\section{Project Estimation}
		\section{Development Process}
			\subsection{Debugging}
		\section{Application Improvements}
			\subsection{Web to Native Application Communication}
		\section{Conclusion}
	
	\appendix
	\chapter{}
	\begin{figure}[h!]
		\caption{Typing in Java}
		\begin{lstlisting}[language=Java,frame=single,breaklines=true]
socket.on("offer", new Emitter.Listener() {
	@Override
	public void call(Object... args) {
		
	}
})
		\end{lstlisting}
	\end{figure}
\end{document}          
